%-----------主文档 格式定义---------------------------------
\addtolength{\headsep}{-0.1cm}        %页眉位置
%\CTEXsetup[name={第,讲},number={\chinese{chapter}}]{chapter}
\titleformat{\chapter}{\centering\Huge\bfseries}{第\,\thechapter\,讲}{1em}{}
%\CTEXsetup[nameformat+={\Large},titleformat+={\Large}]{chapter}
%\CTEXsetup[beforeskip={-23pt},afterskip={20pt plus 2pt minus 2pt}]{chapter}

%\CTEXsetup[format={\bf\flushleft}]{section}
%\CTEXsetup[nameformat+={\large},titleformat+={\large}]{section}
\setcounter{secnumdepth}{4}
%\setcounter{chapter}{+7}
%\titleformat{\chapter}{\centering\Huge\bfseries}{第\,\chinese{chapter}\,讲}{1em}{}
\renewcommand\thesection{\arabic{section}}
\titleformat{\section}{\centering\zihao{3}\bfseries}{第\,\chinese{section}\,讲}{1em}{}
%\titleformat{\section}{\zihao{3}\bfseries}{\chinese{section}、}{0em}{}
\titleformat{\subsection}{\zihao{4}\bfseries}{\chinese{subsection}、}{0em}{}
%----------定义、定理环境------------------------
\newcounter{entry}
\counterwithin*{entry}{section}

%\newcommand{\entry}{\vskip1ex\refstepcounter{entry}\noindent$\text{\ding{226}}{\makebox%
%[10pt]{\footnotesize\bf\arabic{chapter}.\arabic{entry}}\hspace{5pt}}$}

\newcommand{\entry}{\vskip1ex\refstepcounter{entry}$\text{\ding{228}}{\makebox{\bf\arabic{section}.\arabic{entry}}\hspace{5pt}}$}

\newcounter{example}
\counterwithin*{example}{section}
\newcommand{\example}{\vskip1ex\refstepcounter{entry}\refstepcounter{example}$\text{\ding{228}}{\makebox{\bf\arabic{section}.\arabic{entry}\heiti 例\arabic{example}}\hspace{5pt}}$}

\newcounter{definition}
\counterwithin*{definition}{section}
\newcommand{\definition}{\vskip1ex\refstepcounter{entry}\refstepcounter{definition}$\text{\ding{228}}{\makebox{\bf\arabic{section}.\arabic{entry}\heiti 定义\arabic{definition}}\hspace{5pt}}$}

\newcounter{theorem}
\counterwithin*{theorem}{section}
\newcommand{\theorem}{\vskip1ex\refstepcounter{entry}\refstepcounter{theorem}$\text{\ding{228}}{\makebox{\bf\arabic{section}.\arabic{entry}\heiti 定理\arabic{theorem}}\hspace{5pt}}$}

\newcounter{proposition}
\counterwithin*{proposition}{section}
\newcommand{\proposition}{\vskip1ex\refstepcounter{entry}\refstepcounter{proposition}$\text{\ding{228}}{\makebox{\bf\arabic{section}.\arabic{entry}\heiti 命题\arabic{proposition}}\hspace{5pt}}$}

\newcounter{property}
\counterwithin*{property}{section}
\newcommand{\property}{\vskip1ex\refstepcounter{entry}\refstepcounter{property}$\text{\ding{228}}{\makebox{\bf\arabic{section}.\arabic{entry}\heiti 性质\arabic{property}}\hspace{5pt}}$}

\newcounter{lemma}
\counterwithin*{lemma}{section}
\newcommand{\lemma}{\vskip1ex\refstepcounter{entry}\refstepcounter{lemma}$\text{\ding{228}}{\makebox{\bf\arabic{section}.\arabic{entry}\heiti 引理\arabic{lemma}}\hspace{5pt}}$}

\newcounter{corollary}
\counterwithin*{corollary}{theorem}
\newcommand{\corollary}{\vskip1ex\refstepcounter{entry}\refstepcounter{corollary}$\text{\ding{228}}{\makebox{\bf\arabic{section}.\arabic{entry}\heiti 推论\arabic{corollary}}\hspace{5pt}}$}

\newcounter{remark}
\counterwithin*{remark}{section}
\newcommand{\remark}{\vskip1ex\refstepcounter{entry}\refstepcounter{remark}$\text{\ding{228}}{\makebox{\bf\arabic{section}.\arabic{entry}\heiti 注}\hspace{5pt}}$}

\numberwithin{equation}{section}
\renewcommand{\theequation}{\arabic{section}.\arabic{equation}}
%\numberwithin{equation}{section}
%\newcounter{Lem}[chapter]
%\renewcommand{\theLem}{\thechapter.\arabic{Lem}}
%\newenvironment{lemma}{\par{\heiti 引理\stepcounter{Lem}\theLem}\hspace{1ex}}{\par}

%\newcounter{Coro}[chapter]
%\renewcommand{\theCoro}{\thechapter.\arabic{Coro}}

%------------------------------------------
%\newenvironment{proof}{\par{\heiti{证明:}}}{\hfill $\square$ \par}
%p\newenvironment{solution}{\par{\heiti{解:}}}{\hfill $\square$ \par}

\newcommand{\exercise}[1]{\noindent\textbf{练习#1}}
\newcommand{\note}{\noindent\textbf{注记}\ }
\newcommand{\solve}{\textbf{解:}\ }
\newcommand{\proof}{\textbf{证明:}\ }
\newcommand{\analysis}{\noindent\textbf{分析}\ }


% Matlab highlight color settings
%\definecolor{mBasic}{RGB}{248,248,242}       % default
\definecolor{mKeyword}{RGB}{0,0,255}          % bule
\definecolor{mString}{RGB}{160,32,240}        % purple
\definecolor{mComment}{RGB}{34,139,34}        % green
\definecolor{mBackground}{RGB}{245,245,245}   % lightgrey
\definecolor{mNumber}{RGB}{128,128,128}       % gray

% Matlab highlight color settings
%\definecolor{mBasic}{RGB}{248,248,242}       % default
\definecolor{mKeyword}{RGB}{0,0,255}          % bule
\definecolor{mString}{RGB}{160,32,240}        % purple
\definecolor{mComment}{RGB}{34,139,34}        % green
\definecolor{mBackground}{RGB}{245,245,245}   % lightgrey
%\definecolor{mNumber}{RGB}{128,128,128}       % gray
\definecolor{mNumber}{RGB}{134,145,148}       % gray
\definecolor{mNumberbg}{RGB}{237,240,241}     % lightgrey

% Python highlight color settings
%\definecolor{pBasic}{RGB}{248, 248, 242}     % default
\definecolor{pKeyword}{RGB}{228,0,128}        % magenta
\definecolor{pString}{RGB}{148,0,209}         % purple
\definecolor{pComment}{RGB}{117,113,94}       % gray
\definecolor{pIdentifier}{RGB}{166, 226, 46}  %
\definecolor{pBackground}{RGB}{245,245,245}   % lightgrey
\definecolor{pNumber}{RGB}{128,128,128}       % gray

\lstnewenvironment{Matlab}[1]{
	\lstset{language=Matlab,               % choose the language of the code
		%frame=tlbr,
		xleftmargin=10pt,
		xrightmargin=10pt,
		frame=l,
		framesep=15pt,%framerule=0pt,  % sets the frame style
		%frame=shadowbox,rulesepcolor=\color{red!20!green!20!blue!20},
		basicstyle=\small\ttfamily,
		keywordstyle={\color{mKeyword}},     % sets color for keywords
		stringstyle={\color{mString}},       % sets color for strings
		commentstyle={\color{mComment}},     % sets color for comments
		backgroundcolor=\color{mBackground}, % choose the background color
		%title=#1,                            % \lstname show the filename of files
		keywords={break,case,catch,classdef,continue,else,elseif,end,for,
			function,global,if,otherwise,parfor,persistent,return,spmd,switch,try,while},
		showspaces=false,                    % show spaces adding particular underscores
		showstringspaces=false,              % underline spaces within strings
		showtabs=false,                      % show tabs within strings adding particular underscores
		tabsize=4,                           % sets default tabsize to 2 spaces
		%captionpos=t,                        % sets the caption-position to bottom
		%breaklines=true,                     % sets automatic line breaking
		framexleftmargin=5pt,
		fillcolor=\color{mNumberbg},
		rulecolor=\color{mNumberbg},
		numberstyle=\tiny\color{mNumber},
		numbersep=9pt,                      % how far the line-numbers are from the code
		numbers=left,                        % where to put the line-numbers
		stepnumber=1,                        % the step between two line-numbers.
}}{\lstname}

\definecolor{dkgreen}{rgb}{0,0.6,0}
\definecolor{gray}{rgb}{0.5,0.5,0.5}
\definecolor{mauve}{rgb}{0.58,0,0.82}
\lstset{frame=tb,
	language=Matlab,
	aboveskip=3mm,
	belowskip=3mm,
	xleftmargin=10pt,
	xrightmargin=10pt,
	frame=l,
	framesep=15pt,%framerule=0pt,  % sets the frame style
	%frame=shadowbox,rulesepcolor=\color{red!20!green!20!blue!20},
	basicstyle=\small\ttfamily,
	keywordstyle={\color{mKeyword}},     % sets color for keywords
	stringstyle={\color{mString}},       % sets color for strings
	commentstyle={\color{mComment}},     % sets color for comments
	backgroundcolor=\color{mBackground}, % choose the background color
	showstringspaces=false,
	columns=flexible,
	backgroundcolor=\color{mBackground}, % choose the background color
	numbers=left,%设置行号位置none不显示行号
	keywords={break,case,catch,classdef,continue,else,elseif,end,for,
		function,global,if,otherwise,parfor,persistent,return,spmd,switch,try,while},
	showspaces=false,                    % show spaces adding particular underscores
	showstringspaces=false,              % underline spaces within strings
	showtabs=false,                      % show tabs within strings adding particular underscores
	tabsize=4,                           % sets default tabsize to 2 spaces
	%numberstyle=\tiny\courier, %设置行号大小
	numberstyle=\tiny\color{gray},
	keywordstyle=\color{blue},
	commentstyle=\color{dkgreen},
	stringstyle=\color{mauve},
	breaklines=true,
	breakatwhitespace=true,
	escapeinside=``,%逃逸字符(1左面的键),用于显示中文例如在代码中`中文...`
	tabsize=4,
	extendedchars=false %解决代码跨页时,章节标题,页眉等汉字不显示的问题
	framexleftmargin=5pt,
	fillcolor=\color{mNumberbg},
	rulecolor=\color{mNumberbg},
	numberstyle=\tiny\color{mNumber},
	numbersep=9pt,                      % how far the line-numbers are from the code
	numbers=left,                        % where to put the line-numbers
	stepnumber=1,                        % the step between two line-numbers.
}

%\newcommand{\exercise}[1]{\noindent\tcbox[on line,top=0mm,bottom=0mm,%
%	right=0mm,left=0mm]{\bfseries 练习#1}\ }
%\newcommand{\note}{\noindent\textbf{注记}\ }
%\newcommand{\solve}{\noindent\textbf{解}\ }
%\newcommand{\analysis}{\noindent\textbf{分析}\ }
%-------------定义页眉下单隔线----------------
\newcommand{\makeheadrule}{\makebox[0pt][l]{\rule[.7\baselineskip]{\headwidth}{0.3pt}}\vskip-.8\baselineskip}
%-------------定义页眉下双隔线----------------

%用tikz画边框	

\def\Obiankuang{\leavevmode\vbox to0pt{
		\vss\rlap{\hskip 0cm
			\tikz \draw(16.6,0)--(0,0)--(0,-25.6)--(16.6,-25.6)--(16.6,0)--(13.6,0)--(13.6,-25.6);		
		}\vskip -25.7cm}}

\def\Ebiankuang{\leavevmode\vbox to0pt{
		\vss\rlap{\hskip -3cm
			\tikz \draw(16.6,0)--(0,0)--(0,-25.6)--(16.6,-25.6)--(16.6,0)--(3,0)--(3,-25.6);		
		}\vskip -25.7cm}}
%\reversemarginpar
%\fancyhead{} 
%\chead{}
%\lhead[E]{\boxhack \boxhackb } %边框 
%\lhead[O]{\boxhack \boxhackb } %边框 
%\lhead[E]{\lbiankuang}%边框 
%\rhead[O]{\rbiankuang}%边框 

\makeatletter
%\renewcommand{\headrule}{{\if@fancyplain\let\headrulewidth\plainheadrulewidth\fi\makeheadrule}}
\pagestyle{fancy}
\renewcommand{\sectionmark}[1]{\markright{\thesection\quad #1}{}}    %去掉节标题中的点
\fancyhf{} %清空页眉
\renewcommand\headrulewidth{0pt}
\fancyfoot[C]{\kaishu{\footnotesize-~\thepage~-}}         % 页码显示左边
\fancyhead[LO]{\Obiankuang} % 奇数页码中间显示节标题
\fancyhead[LE]{\Ebiankuang}  % 偶数页码中间显示章标题

%\lhead[O]{\Obiankuang}%边框
%\lhead[E]{\Ebiankuang}%边框

%----------------------------------------------------------
\setlength{\parindent}{2em}
